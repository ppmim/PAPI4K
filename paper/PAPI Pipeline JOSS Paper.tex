\documentclass[10pt]{article}
\usepackage[utf8]{inputenc}
\usepackage{amsmath}
\usepackage{hyperref}
\usepackage{natbib}
\usepackage{geometry}
\geometry{a4paper, margin=1in}
\usepackage{enumitem}

\title{PAPI: An Open-Source Data Reduction Pipeline for the PANIC Instrument at Calar Alto Observatory}

\author{
  José-Miguel Ibáñez-Mengual$^{1}$\thanks{\texttt{jmiguel@iaa.es}}, 
  Matilde Fernández$^{1}$
}
\date{\today}

\begin{document}

\maketitle

\begin{center}
  $^1$Instituto de Astrofísica de Andalucía (IAA-CSIC), Granada, Spain
\end{center}

\section*{Summary}

The PANIC Pipeline (PAPI) is an open-source Python-based software package designed for the automated reduction of near-infrared imaging data from the PAnoramic Near-Infrared Camera (PANIC) at the Calar Alto Observatory (CAHA). Initially developed for the HAWAII-2RG detectors of PANIC, PAPI has been updated to support the HAWAII-4RG detector introduced in 2025. It provides a comprehensive suite of tools for processing raw astronomical images, including basic calibration, cosmic-ray removal, crosstalk correction, sky subtraction, non-linearity correction, and astrometric registration. PAPI also includes the PANIC Quick-Look Tool (PQL), a graphical user interface for real-time data quality assessment during observations. Available under the GNU General Public License, PAPI is a versatile tool optimized for broadband imaging of extragalactic sources, such as galaxy surveys and cluster studies, and is adaptable to data from other instruments like Omega2000 and HAWK-I. \cite{ibanez2010panic}

\section*{Statement of Need}

Astronomical data reduction pipelines are critical for processing raw observational data into science-ready formats. The PANIC instrument, a near-infrared camera installed at CAHA’s 2.2m and 3.5m telescopes, generates complex datasets requiring specialized processing to account for instrumental effects such as detector non-linearity, crosstalk, and field distortions. Although there is general purpose astronomical software, PAPI addresses the specific needs of PANIC data, offering an automated, user-friendly pipeline that integrates seamlessly with the observatory’s Observing Tool (OT) and GEIRS scripts. Its open-source nature and Python 3 implementation ensure accessibility, extensibility, and compatibility with modern astronomical workflows. PAPI fills a gap for astronomers requiring efficient, high-quality data reduction for near-infrared imaging, particularly for extragalactic research.

\section*{Functionality}

PAPI automates the reduction of PANIC data through a modular architecture, with key processing steps including:

\begin{itemize}
  \item \textbf{Basic Calibration}: Removal of instrumental signatures using dark and flat-field frames.
  \item \textbf{Cosmic-Ray Removal}: Robust detection and correction of cosmic-ray artifacts.
  \item \textbf{Crosstalk Correction}: Mitigation of electronic ghosting in the HAWAII-2RG and HAWAII-4RG detectors.
  \item \textbf{Sky Subtraction}: Removal of background-sky emission for an improved signal-to-noise ratio.
  \item \textbf{Non-Linearity Correction}: Correction of count-rate-dependent nonlinearity using polynomial coefficients.
  \item \textbf{Astrometric Registration}: Alignment and mosaicking using Astrometry.net and SCAMP for precise World Coordinate System (WCS) calibration.
  \item \textbf{Photometric Calibration}: Support for 2MASS-based photometry with an accuracy ranging from 0.01 to 0.1 magnitudes, depending on field star brightness. \cite{cardenas2017panic}
\end{itemize}

The pipeline supports both command-line execution and integration with PQL, which provides real-time visualization and quality control via a Qt-based interface. PAPI processes FITS files with 32-bit integer raw data and outputs 32-bit floating-point reduced images, handling both single-frame and multi-extension (MEF) formats. Configuration is managed through a flexible \texttt{papi.cfg} file, allowing users to customize parameters such as input/output directories, calibration settings, and processing options. \cite{papi_docs}

\section*{Implementation}

PAPI is implemented in Python 3.7+ and relies on dependencies such as Astropy, NumPy, and SCAMP, with optional support for Montage and DS9 for visualization. It is developed and tested on 64-bit Linux systems (openSUSE 15.4 and Ubuntu 19.1), with installation facilitated through Anaconda or virtualenv. The pipeline integrates external tools like SWARP for image co-addition and Astrometry.net for astrometric calibration, wrapped in Python for seamless operation. The codebase is hosted on GitHub, with documentation built using Sphinx, available in HTML and PDF formats. The modular design of PAPI allows for the standalone execution of individual modules (e.g., to create master calibration frames) or full pipeline processing via the \texttt{run_papi.py} script. \cite{papi_docs}

\section*{Usage and Community}

PAPI is primarily designed for PANIC data but is adaptable to other near-infrared instruments like Omega2000 and HAWK-I, though not fully optimized for them. It supports both automated reduction of the observation blocks defined by CAHA’s OT and manual processing through GEIRS scripts. The PQL tool enhances observational efficiency by enabling real-time monitoring and quality checks, displaying results in DS9 and matplotlib. The open source license encourages community contributions, with bug reports and feature requests managed via the GitHub issue tracker. Recent updates (e.g., version 3.0.0) include support for the HAWAII-4RG detector and compatibility with modern Linux distributions. \cite{papi_docs}

\section*{Acknowledgements}

We acknowledge support from the Instituto de Astrofísica de Andalucía (IAA-CSIC) and the Calar Alto Observatory. The development was partially funded by projects related to the PANIC instrument. The authors thank the astronomical community for feedback and contributions via GitHub.

\bibliographystyle{plain}
\bibliography{references}

\end{document}

\begin{thebibliography}{9}
\bibitem{ibanez2010panic}
Ibáñez Mengual, J.M., Fernández, M., Rodríguez Gómez, J.R., García Segura, A.J., Storz, C., “The PANIC software system”, \emph{Proc. SPIE}, 7740, 77402E, 2010.

\bibitem{cardenas2017panic}
Cárdenas Vázquez, M.C., et al., “PANIC: A General-purpose Panoramic Near-infrared Camera for the Calar Alto Observatory”, \emph{Publications of the Astronomical Society of the Pacific}, 130, 2017.

\bibitem{papi_docs}
Ibáñez-Mengual, J.M., “PAPI 2.3.20250508083928 documentation”, \url{https://home.iaa.csic.es/~jmiguel/PANIC/PAPI/html/index.html}, 2025.
\end{thebibliography}